\documentclass[10pt,a4paper]{article}
\usepackage[utf8]{inputenc}
\usepackage{amsmath}
\usepackage{amsfonts}
\usepackage{amssymb}
\usepackage{textcomp}
\begin{document}

\section{Introduction}
\subsection{Basics}
Invariant sets: $A\subseteq U,$ s.t. $A\cdot t=A$\\

Long time behavior: $\omega(x)=\{y|x\cdot t_n\to y\text{ for some }t_n\to\infty\}$\\

Stability or attraction: \\
Fixed point $p$ is stable, if $\forall\epsilon>0,\exists\delta>0,$ s.t. $B_{\delta}(p)\cdot t\subset B_{\epsilon}(p),\forall t\geqslant 0$\\

Fixed point $p$ is attracting, if (1) stable, (2) $\exists\epsilon >0,x\in B_{\epsilon}(p)\Longrightarrow\omega(x)=\{p\}$\\

Ex. Stable: surrounded by orbits, not coming in. Attracting: actually goes to the point. Unstable: like saddle points.\\

Exercise: work out definitions for periodic orbit and invariant sets in general.\\

\subsubsection{Dimension} 
\begin{itemize}
	\item in 1, just need to figure out FP's and the signs of the flows.\\
	\item in 2, organized by FP and periodic orbits. Poincare-Bendixson theorem: $\omega(x)\text{ compact }\Longrightarrow\omega(x)\text{ contains a FP, or is a periodic orbit.}$
	\item Chaotic dynamics: near trajectories separated after certain time
\end{itemize}

\subsection{Linearization}
$p$ FP, to understand the dynamics near $p$
\begin{itemize}
	\item step1: Linearization of $p$, i.e. $Df(p)$, linear system: $\dot{y}=Df(p)y$, dynamics of this one tells something of the original.
	\item Question: to what extent?
	\item dynamics of linear system: set $A=Df(p)$, spectrum of $A$, $\sigma(A)=$eigenvalues of $A$, we have $n$ eigenvalues counting multiplicity.
	\item decomposition of $\sigma(A)=\sigma_{-}(A)\bigcup\sigma_{0}(A)\bigcup\sigma_{+}(A)$, decompose $\mathbb{R}^n=E_{-}\bigoplus E_{0}\bigoplus E_{+}$, each invariant under the corresponding spectrum.
	\item $\sigma_{-}(A)$ part: exponentially decay, $\sigma_{+}(A)$ part: exponentially decay for $t\to-\infty$, i.e. exponential growth, $\sigma_{0}(A)$ part: undetermined.
	\item $E_{-}$ stable subspace, $E_{0}$ center subspace, $E_{+}$ unstable subspace
\end{itemize}

\subsection{Invariant Manifold}
Dynamics near a FP, $p,f(p)=0,\dot{y}=Df(p)y,A=Df(p),\sigma(A)$\\
Goes to nonlinearity locally, "curved" versions of $E_{-/0/+}$, $W_{loc}^{s/c/u}$, invariant under the flow relative to a neighborhood of $p$.\\
Graph for respective subspace to a complement subspace; each is tangent to their subspace \\
\begin{itemize}
	\item $W_{loc}^s$ trajectory decays to $p$ exponentially $t\to\infty$
	\item $W_{loc}^u$ trajectory decays to $p$ exponentially $t\to -\infty$
	\item $W_{loc}^c$ trajectory that does neither
	\item stable/unstable manifolds are determined uniquely on exponential decay condition
\end{itemize}
Global: $W^s=\bigcup_{t\leq 0}W_{loc}^s\cdot t$, the other two similarly\\

\subsection{Bifurcation Theory}
Question: How does dynamics change as parameter varies.\\
$\dot{x}=f(x,\mu),x\in\mathbb{R}^n,\mu\in\mathbb{R}$

\subsubsection{Basic Bifurcations}
\begin{itemize}
	\item n=1
	\begin{itemize}
		\item saddle-node: stable+unstable \textrightarrow  combine to unstable \textrightarrow  disappear
		\item transcritical bifurcation: stable+unstable \textrightarrow  combine to unstable \textrightarrow  unstable+stable
		\item pitchfork bifurcation: one splits to 3
	\end{itemize}
	\item n=2
	\begin{itemize}
		\item Hopf: attracting fp, then attracting periodic orbit, and the fp becomes unstable
	\end{itemize}
	\item Note:
	\begin{itemize}
		\item the pitchfork and the Hopf described above are supercritical, we can also have subcritical
		\item can happen in higher dimensions on a center manifold
	\end{itemize}
	
\end{itemize}

\end{document}