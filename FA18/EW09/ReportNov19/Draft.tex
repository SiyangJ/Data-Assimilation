\documentclass[10pt]{article}
\usepackage{amsmath}
\usepackage{amsfonts}
\usepackage[utf8]{inputenc}

\begin{document}

\section{The Larger DA Problem}

An accurate estimate of the sea ice state is very important. Many  previous studies use ice concentration derived from satellite radiance data as observation in ice data assimilation \cite{Andersen06}. \cite{Lisaeter03} assimilates ice concentration retrieved from SSM/I using EnKF, and they found their method improved estimates of ice concentration compared to a free-run of ice-ocean model experiment. \cite{Caya10} and \cite{Stark10} also assimilated retrieved ice concentration with different model settings and data assimilation methods. However, the algorithms used to retrieve ice concentration, for example, the NASA Team 2 (NT2) algorithm \cite{Markus00}, has various problems, on of which is that it does not take into account of the presence of melt ponds, where melt ponds cannot be distinguished from open water. Therefore, we are motivated to directly assimilate the satellite radiance data, instead of assimilating the retrieved ice concentration.

We found an article with similar idea. In \cite{Scott12}, a method for directly assimilating brightness temperatures in a sea ice model was developed and showed slight improvements in the model forecast most notably during the melt season. Their method directly modeled the emissivity of the sea ice with a simple seasonal parameterization and made use of an atmospheric radiative transfer model to map the state vector to the satellite brightness temperature. However, this is computationally expensive and complex because of the radiative transfer model. The assimilation was also sensitive to how the emissivity was modeled. 

We aim to do something similar but, first, we aim to assimilate satellite radiance instead of brightness temperature, and second, we want not to model the observation operator but use machine learning to develop a statistical model from past data. The inputs to our observation operator would be the state variables in the sea ice model and the atmospheric conditions. The output should be satellite radiance. Once the observation operator is learned, the advantage here is that our operator could be independent of what the actual sea ice emissivities are, adaptive to a specific sea ice models state variables, and computationally inexpensive.

Before we prepare the data set for machine learning the observation operator and test our method in a real sea ice model, we want to develop a proxy model to see what challenges would possibly be posed and if our idea would be useful.

\section{Model}

Our system is modified from \cite{Eisenman09} and we make it a Filippov system. Our state variables are $\alpha_m$, the maximum attainable albedo, and $E$, the energy. We also define the albedo (the percent of incoming solar radiation the ice reflects) $\alpha$ to be a function of $\alpha_m$ and $E$:
\begin{align}
\alpha(E,\alpha_m)=\frac{\alpha_{ml}+\alpha_m}{2}+\frac{\alpha_{ml}-\alpha_m}{2}\tanh\left(\frac{E}{L_i h_{c}} \right) \label{alph}
\end{align}
The boundary for Filippov system is defined by
\begin{align}
H(E,\alpha_m)=\alpha(E,\alpha_m)-0.6 \label{H}.
\end{align}
Separated by $H(E,\alpha_m)=0$, our two sets of dynamics are given by
{\small
\begin{align}
&\frac{dE}{dt}=[1-\alpha(E,\alpha_{m})]F_s(t)-F_0(t)+F_{co_2}-F_T(t)\frac{E}{c_{ml} H_{ml}}+F_B & \label{ew}\\
&\frac{d \alpha_{m}}{dt}= \frac{E^2}{K^2}\alpha_{m}\left(1-\frac{\alpha_{m}}{0.8}\right) + \frac{K^2}{1+E^2}\alpha_{m}\left(1-\frac{\alpha_{m}}{0.6}\right) \quad \textrm{if} \quad  H(E,\alpha_m)>0 &\label{coldalph} \\
&\frac{ d \alpha_{m}}{dt}=\frac{K^2}{1+E^2}\alpha_{m}\left(1-\frac{\alpha_{m}}{0.6}\right) +\frac{E^2}{K^2}\alpha_{m}\left(1-\frac{\alpha_{m}}{0.2}\right) \quad \textrm{if}  \quad H(E,\alpha_m)<0 &\label{warmalph}
\end{align}}

In equation (\ref{ew}), $F_s(t)$ is the incoming solar radiation, $F_0(t)$ is the amount of long wave radiation(heat) that escapes to space $F_{co_2}$ is the amount of long wave radiation reflected back from clouds and $co_2$ , $F_T(t)$ has to do with heat exchange with lower latitudes, $F_B$ is the heat input from the ocean below the ice. While these terms are written to be time dependent we can take time averaged values to make the system autonomous. Equation (\ref{alph}) represents the albedo (the percent of incoming solar radiation the ice reflects) and depends on energy of the ice as a whole. In this equation $\alpha_{ml}$ is the albedo of the ocean mixed layer, $L_i$ the latent heat of fusion for ice, and $h_c$ a chosen characteristic ice thickness which is used to control the smoothness of the parameterization. 
 
In very cold conditions the maximum attainable albedo of the surface should it tend toward $0.8$ with snow fall and other processes keeping the albedo high. When the ice is in warmer conditions, like melting, the maximum attainable albedo should tend to something lower and there should be a competition of values. Initially as the ice begins to pond the albedo drives down pretty quickly, however as the ice temperature increases the melt ponds drain out and the albedo of the ice recovers very quickly. What we will want here is for energies away from zero, but warming conditions, for the albedo to tend to something like say $0.2$, but closer to $E=0$, when the ice is permeable and the ponds can drain exposing the ice surface, the maximum attainable albedo should tend toward that of bare ice $0.6$.

In order to accomplish this we will model the rates of change of the maximum attainable albedo with logistic models setting the carrying capacity to be $0.8$ (eq. \ref{coldalph}) in ``cold" conditions and using competing models with carrying capacities of $0.2$ and  $0.6$ (eq. \ref{warmalph}) in ``warm" conditions. We also let the growth or decay rate in these equations depend on the energy $E$ and some scaling factor $K$. In cold conditions we want the maximum attainable albedo $\alpha_m$ to approach $0.8$ rapidly when the energy is largely negative. For this reason the growth rate is taken with the energy in the numerator (eq. \ref{coldalph}). In warm conditions we wish the rate that $\alpha_m$ approaches $0.2$ to be faster when the energy is away from zero but to approach $0.6$ faster when the energy is near 0.  As a result, we take the energy to be in the denominator with the addition of 1 to avoid singularities at $E=0$ (eq \ref{warmalph}) for the logistic model with a carrying capacity of $0.6$ and the energy in the numerator for the logistic model with a carrying capacity of $0.2$ . 

As for the discontinuity boundary of the Filippov system (eq \ref{H}), we simply define it to be where the albedo of the system $\alpha(E,\alpha_m)$ crosses the $\alpha=0.6$ threshold. We found a matlab program {\it disode45} \cite{Calvo16} that handles the Filippov system automatically.

For our particular problem we will need to define some kind of ``concentration" from our state variables. We will also need to define a non-unique satellite seen concentration and a proxy for the measured satellite radiances.
\begin{align}
&\textrm{ ``Ice concentration":} \hspace{0.1cm}  C_i= \left( \frac{1}{2} + \frac{1}{2}\tanh(\frac{E}{200})\right) \frac{\alpha(E,\alpha_m)}{\alpha_m} & \\
&\textrm{``Pond concentration":} \hspace{0.1cm} C_p= C_i\text{max}\left(0,\frac{( 0.6- \alpha(E,\alpha_m))}{0.6} \right) &\\
&\textrm{``Satellite Radiances":} \hspace{0.1cm}
\begin{bmatrix}
|E\alpha_m|\\ 
\alpha_m-\alpha(E,\alpha_m) \\
\alpha(E,\alpha_m)|E|\\
(0.5+0.4\tanh(\frac{50-E}{10}))(E+273.15)\\
C_i C_p
\end{bmatrix}&\\
&\textrm{``Satellite retrieval concentration":}\hspace{0.1cm} C_{sat}=C_i-C_p &
\end{align}

Here we are defining the concentration of the ice $C_i$ to be the current albedo divided by the maximum attainable surface albedo at time $t$ multiplied by a smoother that forces the concentration to go to zero as the maximum attainable
albedo approaches $0.2$. We take the area fraction of ponds $C_p$ on the surface of any ice to be either $0$ or the concentration of ice times relative difference between the albedo of bare ice and the albedo at time $t$ when the albedo is below that of bare ice, $0.6$. For the satellite radiances, functions of the state variables are chosen to give non-unique results around $E=0$, the energy where ponds form and drain. $C_{sat}$ is a proxy for the ice concentration retrieved from satellite radiance, in this case it will just be the (true) ice concentration minus the concentration of the ponds on the ice surface, since the ponds obscure the ice.

\bibliography{Reference}
\bibliographystyle{ieeetr}


\end{document}